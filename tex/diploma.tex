\documentclass[a4paper,14pt]{extarticle}

% кодировка
\usepackage[utf8]{inputenc}
\usepackage[T2A]{fontenc}

% поля
\usepackage[left=30mm,right=15mm,top=20mm,bottom=20mm]{geometry}

% переносы слов
\usepackage[english,russian]{babel}

% шрифт Таймс
\usepackage{tempora}
\usepackage{newtxmath}

% межстрочный интервал
\usepackage[onehalfspacing]{setspace}

% интервал между абзацами (не регулируется ГОСТом)
\setlength{\parskip}{0.15em}

% отступ первой строки
\usepackage{indentfirst}
\setlength{\parindent}{1.25cm}

% скрытый структурный элемент
\newcommand{\hidedstructel}[1]{%
    \clearpage
    \phantomsection
    \section*{#1}%
}

% структурный элемент
\newcommand{\structel}[1]{%
    \hidedstructel{#1}
    \addcontentsline{toc}{section}{#1}%
}

% счетчик приложений
\usepackage{totcount}
\newtotcounter{annexcount}

% приложение
\renewcommand{\thesection}{\Asbuk{section}}
\newcommand{\annex}[1]{%
    \stepcounter{annexcount}%
    \clearpage
    \section{#1}%
}

% оформление структурного элемента и приложения
\usepackage{titlesec}
\titleformat{\section}
    [display]                   % форма
    {\filcenter\bfseries}       % формат полностью
    {ПРИЛОЖЕНИЕ \thesection}    % метка
    {0pt}                       % отступ от метки
    {}                          % код перед телом

% раздел
\newcommand{\sect}[1]{%
    \clearpage
    \setcounter{figure}{0}  % сбросить нумерацию внутри раздела
    \setcounter{table}{0}
    \setcounter{listing}{0}
    \subsection{#1}
    \renewcommand{\theparagraph}{\thesubsection.\arabic{paragraph}}
}
\titleformat{\subsection}{\filright\bfseries}{}{0pt}{\thesubsection\hspace{1em}}

% Переменная, которая регулирует отступы до и после заголовков подразделов. ГОСТ не регулирует этот вопрос
\newcommand{\headingsMargin}{0.5em}

\titlespacing*{\subsection}
    {\parindent}      % отступ слева
    {0pt}             % сверху
    {\headingsMargin} % снизу
\renewcommand{\thesubsection}{\arabic{subsection}}

% подраздел
\usepackage{placeins}
\newcommand{\subsect}[1]{%
    \FloatBarrier
    \subsubsection{#1}
    \renewcommand{\theparagraph}{\thesubsubsection.\arabic{paragraph}}
}
\titleformat{\subsubsection}{\filright\bfseries}{}{0pt}{\thesubsubsection\hspace{1em}}

\titlespacing*{\subsubsection}{\parindent}{\headingsMargin}{\headingsMargin}

% пункт
\newcommand{\parag}[1]{
    \paragraph{#1}
}
\titleformat{\paragraph}{\filright\bfseries}{\theparagraph}{1em}{ }  % для отступа
\titlespacing*{\paragraph}{\parindent}{\headingsMargin}{\headingsMargin}

% подпункт
\newcommand{\subparag}{
    \subparagraph{}
}
\titleformat{\subparagraph}[runin]{}{\thesubparagraph}{1em}{ }
\titlespacing*{\subparagraph}{\parindent}{0pt}{0pt}

% содержание
\usepackage{etoc}
\setcounter{tocdepth}{3}

% глубина нумерации разделов
\setcounter{secnumdepth}{5}

% перечисления
\usepackage{enumitem}
\setlist{
    topsep=0pt,                   % отступ сверху и снизу списка
    partopsep=0pt,                % то же самое
    leftmargin=0pt,               % отступ слева
    labelsep=0pt,                 % отступ метки
    align=left,                   % выравнивание метки
    listparindent = \parindent,   % отступ первой строки абзаца
    itemsep = \parskip,           % отступ между элементами
    parsep=0pt                    % отступ между абзацами и элементами
}
\setlist[itemize]{
    label=--~,  % в списках тире короткое, в тексте - длинное
    labelwidth=1.2em,
    itemindent=\parindent+\labelwidth
}
\setlist[enumerate]{
    label=\arabic*),
    labelwidth=1.4em,
    itemindent=\parindent+\labelwidth
}

% перечисление с буквенными метками
\AddEnumerateCounter*{\asbuk}{\c@asbuk}{7}
\newlist{asblist}{enumerate}{2}
\setlist[asblist]{
    label=\asbuk*),
    labelwidth=1.4em,
    itemindent=\parindent+\labelwidth
}

% подписи
\usepackage[singlelinecheck=false]{caption}
\DeclareCaptionLabelSeparator{gost}{~---~}
\captionsetup{labelsep=gost}

% иллюстрация
\newcommand{\fig}[3][1]{
    \begin{figure}[h]
        \centering
        \includegraphics[width=#1\textwidth]{#2}
        \caption{#3}\label{#2}
    \end{figure}
}
\renewcommand{\thefigure}{\thesubsection.\arabic{figure}}
\DeclareCaptionLabelFormat{gostfigure}{Рисунок #2}
\captionsetup[figure]{justification=centering, labelformat=gostfigure, position=bottom}
% font=singlespacing по умолчанию
%skip=-6pt

% листинг
\usepackage[newfloat, cache=false]{minted}
\newcommand{\lst}[2]{
    \begin{listing}[h]
        \centering
        \caption{#2}\label{#1}
        \begin{minipage}[t]{.8\textwidth}
            \inputminted[
                fontsize=\small,
                frame=single,
                breaklines,
                linenos
            ]{text}{#1}
        \end{minipage}
    \end{listing}
}
\renewcommand{\thelisting}{\thesubsection.\arabic{listing}}
\DeclareCaptionLabelFormat{custlisting}{Листинг #2}
\captionsetup[listing]{justification=raggedright, labelformat=custlisting, position=top}

% размер номера строки
\renewcommand{\theFancyVerbLine}{\rmfamily{\small \oldstylenums{\arabic{FancyVerbLine}}}}

% код в документе
\newenvironment{codepiece}[2]
{
    \VerbatimEnvironment
    \begin{listing}[h]
        \centering
        \caption{#2}\label{lst:#1}
        \begin{minipage}[t]{.8\textwidth}
            \begin{minted}[
                fontsize=\small,
                frame=single,
                breaklines,
                linenos
            ]{text}%
}{
            \end{minted}
        \end{minipage}
    \end{listing}
}

% таблица
\newenvironment{tbl}[3]
{
    \begin{table}[h]
        \small
        \centering
        \caption{#2}\label{tbl:#1}
        \begin{tabular}{|#3|}
            \hline
}{
            \hline
        \end{tabular}
    \end{table}
}
\renewcommand{\thetable}{\thesubsection.\arabic{table}}
\DeclareCaptionLabelFormat{gosttable}{Таблица #2}
\captionsetup[table]{justification=raggedright, labelformat=gosttable, position=top}

\usepackage{tabularx}

% объединение строк
\usepackage{multirow}
\newcommand{\mr}[2]{\multirow[t]{#1}{=}{#2}}

% колонки
\usepackage{array}
\newcolumntype{M}[1]{>{\centering\arraybackslash}m{#1}}
\newcolumntype{N}[1]{>{\raggedright\arraybackslash}p{#1}}

% заголовок таблицы
\usepackage{xparse}
\NewExpandableDocumentCommand\thead{t< t> O{1} m m}{%
    \IfBooleanTF{#1}{%
        \IfBooleanTF{#2}{%
            \multicolumn{#3}{|M{#4}|}{#5}%
        }{%
            \multicolumn{#3}{|M{#4}}{#5}%
        }
    }{%
        \IfBooleanTF{#2}{%
            \multicolumn{#3}{M{#4}|}{#5}%
        }{%
            \multicolumn{#3}{M{#4}}{#5}%
        }%
    }%
}

% код в таблице
\newenvironment{tabcode}[1]
{
    \VerbatimEnvironment
    \begin{minipage}[t]{#1\textwidth}
    \begin{minted}[fontsize=\small, breaklines]{text}
}{
    \end{minted}
    \end{minipage}
}

% длинная таблица
\usepackage{longtable}
\newenvironment{longtbl}[3]
{
    \small
    \begin{longtable}{|#3|}
        \caption{#2}\label{tbl:#1}\\
        \hline
}{
        \hline
    \end{longtable}
}

% математика
\usepackage{mathtools}  % amsmath
\numberwithin{equation}{subsection}

% графики
\usepackage{tikz, pgfplots}
\pgfplotsset{compat=newest}

\usepackage{adjustbox}
\usepackage{float}
\usepackage{url}

% источники
\usepackage[
    backend=biber,
    style=numeric-comp,
    bibstyle=gost-numeric,
    language = english,
    autolang = other,
    bibencoding=utf8,
    sorting = none,
]{biblatex}
\addbibresource{bibliography.bib}
\newcommand{\showbib}{%
    \structel{СПИСОК ИСПОЛЬЗОВАННЫХ ИСТОЧНИКОВ}%
    \printbibliography[heading=none]%
}

% отступы в источниках
\defbibenvironment{bibliography}
    {\list
        {}
        {\setlength{\leftmargin}{0pt}%
         \setlength{\itemindent}{\parindent}%
         \setlength{\itemsep}{0pt}%
         \setlength{\parsep}{0pt}}}
    {\endlist}
    {\item
     \printtext[labelnumberwidth]{%
        \printfield{labelprefix}%
        \printfield{labelnumber}%
     }%
     \hspace{0.5em}}

% метка без точки
\DeclareFieldFormat{labelnumberwidth}{#1}

% номер последней страницы
\usepackage{lastpage}

% счетчик источников
\newtotcounter{bibcount}
\AtEveryBibitem{
    \stepcounter{bibcount}%
}

% счетчики таблиц и рисунков
\usepackage{xassoccnt}
\newtotcounter{tblcount}
\DeclareAssociatedCounters{table}{tblcount}
\newtotcounter{figcount}
\DeclareAssociatedCounters{figure}{figcount}

% для отладки
%\usepackage{showframe}
%\renewcommand\ShowFrameLinethickness{0.25pt}
%\renewcommand*\ShowFrameColor{\color{red}}
%\usepackage{graphicx}


\usepackage{microtype}
\usepackage[extdef]{delimset}
\usepackage{csquotes} % иначе babel делает ворнинг
\usepackage{ragged2e}
\usepackage[hidelinks]{hyperref}


\begin{document}

\setcounter{page}{2}

\structel{ВВЕДЕНИЕ}

Симбиотические двойные — это системы, в спектре которых можно выделить линии поглощения, характерные для холодных звёзд, и эмиссионные линии, характерные для горячих туманностей.

Предполагается, что они состоят из близко расположенных красного гиганта и белого карлика, которые традиционно называются холодным и горячим компонентами. Из-за гравитации белого карлика красный гигант перестаёт быть сферически симметрчиным и приобретает каплевидную форму. Кроме того, может происходить перенос массы с красного гиганта на белый карлик, за счёт чего образуется аккреционный диск. Вещество может перетекать с помощью звёздного ветра --- или напрямую с поверхности, если холодный компонент полностью заполняет свою полость Роша.

Поскольку красный гигант не является сферически симметричным, при движении по орбите его блеск меняется. Такая переменность называется эллипоидальной. Помимо чисто геометрического эффекта, на переменность также влияет гравитационное потемнение --- зависимость температуры от ускорения силы тяжести в конкретной точке поверхности звезды. Из-за него полюса красного гиганта будут наиболее горячими и яркими, а «носик», расположенный на экваторе и направленный в сторону точки Лагранжа --- наиболее холодным.

В данной работе анализируются кривые блеска T Северной Короны, измеренные в Крымской обсерватории ГАИШ МГУ в 1996--2003 и 2008-2021. На их основе мы определяем соотношение масс компонентов и наклонение орбиты — то есть решаем обратную задачу.

T Северной Короны --- повторная новая. Она вспыхивала в 1866, 1946 и, предполагается, должна вспыхнуть в 2024 \cite{OutburstAnnounce}. Повторные новые возникают, когда на поверхности белого карлика скапливается достаточно много водорода, перетекшего с красного гиганта, и начинается термоядерная реакция. Светимость в результате повышается на ${\sim} 10$ звёздных величин и медленно снижается в течение десятков дней.

Цель работы --- получить распределение вероятностей наклонения и соотношения масс компонентов двойной звезды T Северной Короны.

Методами исследования являются компьютерное моделирование кривых блеска и статистический вывод. Мы применили байесовский подход, воспользовавшись библиотекой для вероятностного программирования Turing.jl \cite{Turing}.

Вероятностное программирование --- парадигма программирования, предназначенная для работы с вероятностными моделями, которая позволяет автоматизировать статистический вывод и проверку гипотез. Главными идеями вероятностного программирования является автоматизированное применение теоремы Байеса, которое позволяет получить апостериорное распределение вероятностей для параметров задачи, и семплирование из него с помощью марковских цепей. Байесовский подход позволяет легко комбинировать данные от разнородных наблюдений.

Задачами работы являются:
\begin{enumerate}
    \item Численное моделирование физики переменности, получение модельных кривых блеска.
    \item Построение вероятностной модели, учитывающей статистические свойства экспериментальных данных.
    \item Получение апостериорного распределения вероятностей для параметров модели (наклонение, соотношение масс) и семплирование из него.
    \item Анализ распределения, получение оценок параметров и их доверительных интервалов.
\end{enumerate}

Настоящая работа была представлена на конференции МФТИ в 2024.

Новизна исследования заключается в применении байесовского подхода к обратной задаче астрофизики.

Значимость работы заключается в получении оценок параметров двойной звезды T Северной Короны, создании вероятностной модели переменности, в которую можно добавлять реалистичные модели звёздных атмосфер, и которая может быть использована для анализа других симбиотических двойных.


\sect{Физика переменности}

\subsect{Форма полости Роша}

На вещество массой $dm$, находящееся в двойной звёздной системе, действуют три силы: гравитация красного гиганта, гравитация белого карлика и центробежная сила. Область, в которой преобладает гравитация какого-либо из компонентов, называется его полостью Роша. Если вещество оказывается за пределами полости Роша, то оно перетекает на другую звезду или в космическое пространство.

Суммарный потенциал трёх сил равен
\begin{equation}
\Omega = 
-\frac{G m_\text{giant}}{r_1}
-\frac{G m_\text{dwarf}}{r_2}
-\frac{1}{2} (\omega \times r_3)^2
\label{eq:roche_generic}
\end{equation}
где $m_\text{giant}$ и $m_\text{dwarf}$ --- массы двух звёзд, $r_1$ --- расстояние до центра гиганта, $r_2$ --- расстояние до центра карлика, $r_3$ --- расстояние до центра масс, $\omega$ --- угловая скорость вращения звёзд по орбите.

Вещество красного гиганта, находясь под действием потенциала $\Omega$, заполняет некоторую эквипотенциальную поверхность, на которой $\Omega = \text{const}$. Одной из эквипотенциальных поверхностей является полость Роша. Она проходит через точку Лагранжа $L_1$, где все силы уравновешиваются и $\nabla \Omega = 0$.

В уравнении \eqref{eq:roche_generic} неявно предполагается, что гравитационный потенциал можно заменить потенциалом точечного тела. Это приближение оправданно, поскольку внешняя оболочка красных гигантов имеет очень низкую плотность, и масса в основном сосредоточена в практически сферическом ядре.

Найдём форму полости Роша численно. Для этого введём систему координат с центром в ядре красного гиганта, и направим ось $x$ в сторону карлика. Обозначим расстояние между центрами звёзд как $a$. Тогда \eqref{eq:roche_generic} примет вид
\[
\Omega =
-\frac{G m_\text{giant}}{\sqrt{x^2 + y^2 + z^2}}
-\frac{G m_\text{dwarf}}{\sqrt{(x - a)^2 + y^2 + z^2}}
-\frac{1}{2} \omega^2 \brk*{\brk*{x - a \frac{m_\text{dwarf}}{m_\text{giant} + m_\text{dwarf}}}^2 + y^2}
\]

Если обезразмерить координаты $(r \to r / a)$, обозначить соотношение масс $m_\text{giant} / m_\text{dwarf} = q$ и заметить, что в силу третьего закона Кеплера $\omega^2 = \dfrac{G}{a^3} \brk!{m_\text{giant} + m_\text{dwarf}}$, то получится
\begin{align*}
\Omega 
&= - \frac{G m_\text{giant}}{a} \brk[s]*{
    \frac{1}{r}
    + \frac{q^{-1}}{\sqrt{1 + r^2 - 2x}}
    + \frac{1 + q^{-1}}{2} \brk*{\brk*{x - \frac{1}{1+q}}^2 + y^2}
} \\
&= - \frac{G m_\text{giant}}{a} \brk[s]*{
    \frac{1}{r}
    + q^{-1} \brk*{
        \frac{1}{\sqrt{1 + r^2 - 2x}} - x
    }
    + \frac{1 + q^{-1}}{2} \brk!{x^2 + y^2}
    + \frac{q^{-1}}{1 + q}
}
\end{align*}

Последний член является константой и может быть отброшен. Таким образом, безразмерный потенциал равен
\begin{equation}
\Omega = \frac{1}{r} + q^{-1} \brk*{
    \frac{1}{\sqrt{1 + r^2 - 2x}} - x
}
+ \frac{1 + q^{-1}}{2} \brk!{x^2 + y^2}
\label{eq:roche_dekart}
\end{equation}
Или, в сферических координатах,
\begin{equation}
\Omega = \frac{1}{r} + q^{-1} \brk*{
    \frac{1}{\sqrt{1 + r^2 - 2r n_x}} - r n_x
}
+ \frac{1 + q^{-1}}{2} r^2 (1 - n_z^2)
\label{eq:roche_spherical}
\end{equation}
где введены направляющие косинусы $n_x = x / r$, $n_z = z / r$.

Уравнение \eqref{eq:roche_spherical} позволяет численно определить форму полости Роша. Для этого достаточно для всех направлений $(n_x, n_y, n_z)$ решить уравнение $\Omega(r) = \Omega_0.$

В качестве $\Omega_0$ необходимо взять значение потенциала в точке Лагранжа $L_1$. Её положение мы находим, решив уравнение $\dfrac{\partial \Omega(x, 0, 0)}{\partial x} = 0$.


\subsect{Гравитационное потемнение}

Гравитационное потемнение --- явление, при котором температура поверхности звезды зависит от ускорения силы тяжести в данной точке. Обычно его описывают формулой
\begin{equation}
T \sim g^\beta, \quad \beta > 0
\label{eq:gravity_darkening}
\end{equation}

За счёт гравитационного потемнения полюса быстро вращающихся звёзд ярче и горячее, чем экватор. В случае симбиотических двойных гравитационное потемнение тоже оказывает влияние.

Проще всего вывести \eqref{eq:gravity_darkening} для зоны лучистого переноса.

В модели Эддингтона предполагается, что давление внутри звезды складывается из давления газа $P_g$ и давления излучения $P_r = \dfrac{4}{3} \dfrac{\sigma}{c} T^4$, причем их отношение постоянно по всему объему звезды. Тогда суммарное давление можно записать как $P = A P_r$, где $A > 1$ --- некоторая константа.

Градиент давления должен уравновешивать силу тяжести:
\begin{equation}
\nabla P = \rho \vec g = -\rho \nabla \Omega
\label{eq:grads}
\end{equation}

Если рассмотреть смещение вдоль некоторого вектора $e$, то $d\Omega = (e \cdot \nabla \Omega)$,
\[
dP = -\rho d\Omega
\]

В частности, при $d\Omega = 0$ также $dP = 0$. Следовательно, $P$ постоянно на эквипотенциальных поверхностях и является функцией $\Omega$. Более того, $\rho = -\dfrac{dP(\Omega)}{d\Omega}$ тоже является функцией $\Omega$.

Для лучистого переноса верно диффузное приближение: световой поток равен
\begin{equation}
F = -D \nabla u
\label{eq:Fick}
\end{equation}
где $u = \dfrac{4}{c} \sigma T^4$ --- плотность энергии, $D = \dfrac{1}{3} l c$ --- коэффициент диффузии, $l$ --- длина свободного пробега фотонов.

Заметим, что $u = 3 P_r = 3 A^{-1} P$. Это позволяет объединить \eqref{eq:grads} и \eqref{eq:Fick}:
\begin{equation}
\vec F = -3 A^{-1} \rho D \vec g
\label{eq:flux1}
\end{equation}

$l$ и $D$ зависят от плотности, которая является функцией $\Omega$. Поэтому на эквипотенциальной поверхности световой поток пропорционален g:
\[
F \sim g
\]
Впервые этот результат был получен фон Зейпелем ровно 100 лет назад \cite{vonZeipel}. Далее традиционно делается вывод о том, что температура поверхности пропорциональна $g^{1/4}$.

Но это не так. В зоне лучистого переноса $F \neq \sigma T^4$, вместо этого $F$ определяется уравнением \eqref{eq:Fick}. Более того, поскольку давление и плотность постоянны на эквипотенциальной поверхности, то в силу уравнения состояния (каким бы оно ни было), температура тоже постоянна.


\showbib

\end{document}
